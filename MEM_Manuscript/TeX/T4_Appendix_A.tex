\documentclass[T0_MEM]{subfiles}
\linespread{1.1}
\begin{document}

\section{Appendix A -- Accuracy assessment}
\label{sec:appendix_a}

The performance of the six statistical models is assessed based on forecasts of life expectancy. Depending on the model output, the complete remaining life expectancy at age $x$, $e(x)$, can be defined in terms of central death rate $m(x)$,

\begin{equation}
 e(x) = \int_{x}^{\infty} \exp[ -m(s)] ds,
\end{equation}

and with reference to the death frequency $f(x)$ as follows,

\begin{equation}
 e(x) = \int_{x}^{\infty} s f(s) ds \Big/ \int_{x}^{\infty} f(s) ds - x.
\end{equation}

The discrete approximations follow the method presented in \cite{preston2001} and implemented in the \texttt{MortalityLaws} software package.

Let $e(x,t)$ be the expected remaining lifetime of an $x$-year-old calculated using the observed mortality rates at time $t$, and $\hat{e}(x,t)$  the forecast value respectively. Then, for each forecasting-scenario,  model and country a matrix of accuracy errors is obtained,

\begin{equation}
 E = \begin{bmatrix}
    \delta(0,1) & \delta(0,2) & \dots  & \delta(0,\tau) \\
    \delta(1,1) & \delta(1,2) & \dots  & \delta(1,\tau) \\
    \vdots & \vdots & \ddots & \vdots \\
    \delta(\omega,1) & \delta(\omega,2) & \dots  & \delta(\omega,\tau)
\end{bmatrix}
\end{equation}

where $\delta(x,t)$ is the distance between the observed and predicted life expectancy calculated as,

\begin{equation}
\delta(x,t) = e(x,t) - \hat{e}(x,t), \quad x = 0 \dots \omega, \quad t = 1 \dots \tau.
\end{equation}

The aggregate accuracy measures are computed over a forecasting window of 20 years and over the 0-95 age range, i.e $\tau = 20$ and $\omega = 95$. The aggregation process is done by averaging the accuracy results over age and time. To further simplify the notations we define algebraically the \emph{mean} operation of a random variable $z(x,t)$ as:

\begin{equation}
mean\Big[z(x,t)\Big] = \dfrac{\sum_{x=0}^{\omega} \sum_{t=1}^{\tau} z(x,t)}{(\omega + 1)\tau}
\end{equation}

Then the six accuracy measures can be summarized as:

\begin{itemize}
  \item ME -- Mean Error
  \begin{equation}
    ME = mean\Big[\delta(x,t)\Big]
  \end{equation}

  \item MAE -- Mean Absolute Error
  \begin{equation}
  MAE = mean\Big[|\delta(x,t)|\Big]
  \end{equation}

  \item MAPE -- Mean Absolute Percentage Error
  \begin{equation}
  MAPE = mean\Big[ 100 \times \dfrac{|\delta(x,t)|}{e(x,t)} \Big]
  \end{equation}

  \item sMAPE -- Symmetric Mean Absolute Percentage Error
  \begin{equation}
  sMAPE = mean\Big[ 200 \times \dfrac{|\delta(x,t)|}{e(x,t) + \hat{e}(x,t)} \Big]
  \end{equation}

  \item sMRAE -- Symmetric Mean Relative Absolute Error\\
  Let $\delta_B(x,t)$ be the benchmark error. Specifically $\delta_B(x,t)$ is represented by the errors generated by a na\"ive model, here the multivariate random-walk with drift model. Then,

  \begin{equation}
  sMRAE = mean\Big[ 200 \times \dfrac{|\delta(x,t)|}{|\delta(x,t)| + |\delta_B(x,t)|} \Big]
  \end{equation}

  \item MASE -- Mean Absolute Scaled Error
  \begin{equation}
  MASE = mean\Bigg[\Bigg| \dfrac{\delta(x,t)}{ \dfrac{1}{\tau - 1} \sum_{t=2}^{\tau} [e(x,t) - e(x,t-1)] } \Bigg|\Bigg]
  \end{equation}
\end{itemize}


For a specific country and model multiple scenarios of equal dimension (fitted/forecasted years) are explored in order to account for the robustness of the models. Usually, 18 scenarios per country. The aggregation is done by averaging the results over all scenarios for each model/accuracy-measure/country.

\end{document}



























